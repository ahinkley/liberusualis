III. RUBRICS FOR THE MASS

A) GENERAL PRINCIPLES AND RULES.

The holy Sacrifice of Mass, celebrated in accordance with the canons and rubrics, is an act of public worship offered to God in the name of Christ and the Church. This is why the expression "private Mass" should be avoided.

The Mass with the divine Office constitudes the sum of all Christian public worship; the Mass, then, should normally conform to the Office of the day.

However, there also exist Masses which are out of the course of the Office, namely vitive Masses and those of the Dead.

There are two kinds of Masses: {\it Masses in cantu} and {\it said Masses.}

A {\it Mass in cantu} us when the celebrant does in fact sing the parts reserved to him by the rubrice; otherwise it is a {\it said Mass (Missa lecta).}

The {\it Mass in cantu,} moreover, if celebrated with the sacred ministers, is called {\it solemn Mass} or {\it high Mass;} if sung without sacred ministers, {\it Missa cantata.}

{\it High Mass} celebrated by a Bishop, or some other who has the right to do so, with the solemn ceremonies laid down in the liturgical books, is called {\it pontifical Mass.}

Mass by its very nature demands that all who are present should take part in it, each in his proper way.

There are various ways in which the faithful can take an active part in the Sacrifice of Mass. Things must be so ordered, however, that any danger of abuse is removed and that the chief end of this participation may be obtained, which is the enrichment of God and the edification of his people.

This active participation by the faithful is dealt with at length in the {\it Instruction concerning sacred music and the liturgy} issued by the S. Congregation of Rites, 3 Sept. 1958.

B) CONVENTUAL MASS.

Conventual Mass (see the note of p. XXX) should follow Terce, unless the Superior, for a serious reason, should judge that it should follow Sext or None.

On Whitsun Eve conventual Mass follows None.

C) THE VARIOUS PARTS OF THE MASS.

1) {\it The psalm} Iúdica me, Deus, {\it the} Confíteor {\it and the censing of the altar.}

The psalm {\it Iúdica me, Deus} with its antiphon, and the {\it Confíteor} with its absolution are said at the foot of the altar at every Mass whether sung or said; but they are omitted, together with the versibles that follow and the prayers {\it Aufer a nobis} and {\it Orámus te, Dómine,} on the following occasions:
Mass of the Purification of Our Lady that follows the blessing of candles and procession;
the Mass of Ash Wednesday that is sung after the blessing and imposition of ashes;
the Mass of Palm Sunday that follows the blessingof palms and procession;
the Mass of the Paschal Vigil;
the Rogation Mass that follows the procession of the Greater or Lesser Litanies;
the Masses that follow certain consecrations, according to the rubrics in the Roman Pontifical.

The psalm {\it Iúdica me} is omitted also:
in Masses of the season from the I Sunday of the Passion until the Evening Mass on Maundy Thursday;
at Masses of the Dead.

The censings that are obligatory at high Mass are allowed also at all sung Masses.

2) {\it The prayers sung at Mass.}

{\it a)} After the prayer proper to the Mass, a sung Mass {\it other than conventual} allows of no other prayer that but which is said under the same ending as the prayer of the Mass, and a single privileged commemoration (see the list, XXX).

This rule applies to {\it conventual} Mass only on liturgical days of the I class and at votive Masses\footnote{}{For all that concerns votive Masses, see the text of the decree.} of I class.

{\it b)} On II class Sundays no other prayer is allowed beyond the commemoration of a II class feast; but this is omitted if there is a privileged commemoration.

{\it c)} Other II class liturgical days and II class votive Masses only allow of one commemoration, namely one provilefed or one ordinary.

{\it d)} III or IV class liturgical days or votive Masses allow of two additional prayers only.

A prayer that exceeds the number fixed for any liturgical day is omitted; it follows that on no pretext can the number of praters exceed three.

When ever the words {\it Flectámus génua, Leváte,} occur in the Missal, they are said at high Mass by the deacon and at other Masses by the celebrant. Immediately afterwards all kneel and pray with the celebrant for a short space. When {\it Leváte} is said, all rise, and the celebrant says the prayer.

On the last Sunday but one in Octber, or some other Sunday appointed as "Missions Sunday" by the Bishop, the prayer for the Propagation of the Faith is added at all Masses under a single conclusion.

2) {\it The lessons at Mass.}

On Ember Saturdays five lessons precede the Epistle. At conventual Masses and ordination Masses all the lessons must always be read with their versicles and prayers. At other Masses, whether sung or said, only hte first prayer can be said, that which corresponds to the Office, with {\it Flectámus génua} if it is to be said, and the first lesson with the gradual or the Alleluia; then, after the {\it Dóminus vobíscum, Et cum spíritu túo,} and {\it Oremus,} said in the usual way, the second prayer without {\it Flectámus génua,} and any commemorations that may occur. The other lessons that follow are omitted witht their versibles and prayers, and there follows at once the last lesson or epistle with the tract, and on Whit Saturday the sequence.

On an Ember Saturday and the Saturday {\it Sitiéntes} the Mass at which holy orders are conferred should be that of the Saturday, even if a I or II class feast is kept that day.

At sung Mass, everything the deacon, subdeacon or reader sing or read as part of their special function, is omitted by the celebrant.

After the gospel, especially on Sundays and feasts of obligation, a short sermon should be preached, as opportunity allows. But if this is done by a priest other than the celebrant, it should not overlap the celebrant's part and interfere with the people's participation in the Mass; the celebration of Mass must be suspended during the sermon and resumed afterwards.

4) {\it The Creed.}

The Creed is sung:
every Sunday, even if its Office is replaced by another or a II class votive Mass is celebrated;
on I class feasts and in I class votive Masses;
on II class feasts of Our Lord and of Our Lady;
during the Octaves of Christmas, Easter and Whit Sunday, even if an occurrent feast or a votive Mass is being celebrated;
on the {\it dies natalis} of Apostles and Evangelists, and also on the feasts of St. Peter's Chair and of St. Barnabas.

The Creed is not said:
in the Masses of Maundy Thursday and the Easter Vigil;
on II class feasts, except those mentioned above;
in II class votive Masses;
in festive and votive Masses of the III or IV class;
on account of a commemoration made at the Mass;
in Masses of the Dead.

5) {\it The offertory and communiton antiphons.}

These are sometimes followed by {\it allelúia} that forms part of the antiphon itself. In this case this {\it allelúia} is sung throughout the year, except from Septuagesima to Easter;

{\it Ex.} The {\it offertory Beáta es, Vírgo María,} where formerly {\it allelúia} was omitted on July 2 and September 8.

6) {\it Holy Communion.}

The proper time for distributing holy Communion to the people is at Mass immediatelya fter the Communtion of the celebrant. The celebrant should do this himself, except when the number of communicants is so great that he needs to be helped by one or more other priests.

It is absolutely unfitting, that at the altar where Mass is being celebrated, Communion may be given immediately before or after Mass, or even apart from Mass. In this case the form in the Roman Ritual is used.

Whenever holy Communion is given during Mass, the celebrant, having comsumed the Precious Blood, and with no {\it Confíteor} said, omitting the absolution, immediately says {\it Ecce Agnus Dei} and three times {\it Dómine, non sum dignus.} He then distributes holy Communion.

7) {\it The conclusion of Mass.}

At the end of Mass {\it Ite, míssa est} is said, with the answer {\it Déo grátias.} (In Masses XVI, XVII and XVIII of the Kyriale, {\it Ite, míssa est} is taken from Mass XV. This is also allowed in all the Masses in which the {\it Ite, míssa est} is sung without {\it allelúia}).

However:
at the Evening Mass on Maundy Thursday, that is followed by the solemn deposition of the Blessed Sacrament, and at other Masses followed by a procession (for instance on the feast of Corpus Christi), {\it Benedicámus Dómino} is sung, with answer {\it Déo grátias}, as in Mass II or XVI, exclusive of all others.
during the Easter octave, in Masses of the season, a double {\it allelúia} is added to  {\it Ite, míssa est;}
in Masses of the Dead {\it Requiéscant in páce} is said, with answer {\it Amen.}

The celebrant, having said {\it Pláceat,} gives the blessing; which is only omitted when {\it Benedicámus Dómino} or {\it Requiéscant in páce} is said.

As last gospel, at every Mass the beginning of hte gospel according to St. John is said. But on Palm Sunday at all Masses that are not followed by the blessing of palms and procession, the proper last gospel is said.

The last gospel is left out entirely:
at Masses when {\it Benedicámus Dómino} is said;
at the third Christmas Mass;
on Palm Sunday at the Mass that follows the blessing of palms and procession;
at the Mass of the Easter Vigil;
at Masses of the Dead followed by the Absolution at the catafalque;
at Masses that follow certain consecrations, according to the rubrics of the Roman Pontifical.


