\renewcommand{\thesection}{}

\section{INTRODUCTION.}

This new English Edition of the {\it Liber Usualis} gives a brief summary of the Rules for the proper execution and interpretation of the Vatican Edition of the Roman Chant, according to the Solesmes Method.

The Vatican Edition contains the Official Musical Text. {\it To ensure uniformity in the rendering} of the Chant of the Church, ecclesiastical legislation provides that this musical text may be used "with the addition of the Solesmes Rhythmic signs", as an aid.

The use of these signs is officially authorised by the Congregation of Rites. Musicians, generally, have long since experienced the wisdom and even the necessity of this official sanction to the Solesmes Method as the sure means to secure a desired and uniform system of interpretation.

As in all Art-forms, so in Plainsong, rules are the outcome of a wide practical experience, insight and research. The Rules presented here have been worked out and co-ordinated by the Benedictine monks of the Solesmes Congregation. Based as they are on the ancient Manuscript Records, which have been thoroughly examined in their application to the Vatican text, those Rules have for some fifty years proved their efficacy as a convincing guide to the proper unified execution of the Gregorian melodies in the daily carrying out of the Liturgy by the monks of Solesmes themselves.

Our Holy Father, Pope {\sc Pius XI}, in an autograph letter to His Eminence Cardinal Dubois, on the occasion of the Founding of the Gregorian Institute, at Paris, in 1924, writes: {\it "We commend you no less warmly for having secured the services of these same Solesmes monks to teach in the Paris Institute; since, on account of their perfect mastery of the subject, they interpret Gregorian music with a finished perfection which leaves nothing to be desired".}

With this quotation of an august commendation, the present Edition is now offered by the Solesmes monks, that the Roman Chant may be a profitable instrument {\it "capable of raising the mind to God, and better fitted than any other to foster the piety of the Nations".}

This Edition with complete musical notation includes the following:
\begin{enumerate}
\item The Kyriale with {\it Cantus ad libitum}
\item The Mass of the Sundays and Feasts including those of double rank throughout the year, with Vespers and Compline for the same.
\item Prime, Terce, Sext, None, for Sundays and Feasts of the First and Second Class.
\item Matins of Christmas, Easter, Pentecost, Corpus Christi; Lauds for Feasts of the First Class.
\item The Litanies: the Mass of Rogation Days, Ember Days, Easter and Whitsun weeks; the Vigils of Christmas, Epiphany and Whitsun.
\item The services of Ash Wednesday, the Triduum of Holy Week and Easter Day.
\item The principal Votive Masses and the Offices for the Dead.
\end{enumerate}

In the beginning of the book will be found the Common Tones of the Mass and Office. Chants for special occasions, e.g. the Blessing of the Holy Oils, Ordinations, etc, are included in the Appendix.

A practical feature of this work should be noticed and will, it is hoped, be much appreciated: all the Vesper Psalms are set out with the various Tones to which they are sung (see pp 000). The Intonation, Flex, and Cadences are clearly marked for each Psalm. This has not been done for Lauds and the Hours, since these are generally sung by more experienced Choirs.

A small number of Chants for Benediction has been added; the scope of this manual does not allow of a larger number than those in current use. Where greater variety is needed, recourse may be had to special publications and Benediction Manuals already in existence.

The sources of this Compendium are the Missal, the Ritual, the Gradual and the Antiphonary. Recent decisions of the Congregation of Rites have been taken into account. Pieces which have not yet appeared in the Vatican Edition are taken from the approved publications of the Benedictines of Solesmes.

\begin{adjustwidth}{2cm}{}
Voce vita non discordet;\\
Cum vox vita non remordet,\\
Dulce est symphonia. (Adam of St Victor)
\end{adjustwidth}

Feast of St Gregory 1934.

