\section{RUBRICS FOR THE CHANT OF THE MASS.}

I. \quad When the priest goes towards the altar, the cantors begin the Introit. On Ferias and Simples the Intonation is to be sung by one cantor as far as the sign *: on other Feasts and Sundays, there should be two cantors: but on Solemn Feasts there should be four, if as many as four are available. The choir continues until the Psalm. The first part of the Verse of the Psalm as far as the asterisk, and the $\Vbar$. {\it Gloria Patri,} are sung by the cantors, the full choir taking up the rest of the verse. Afterwards, the Introit as far as the Psalm is repeated by the full choir.

If the priest and ministers have some way to go in the church before reaching the altar, there is no reason why several Verses of the Introit Psalm should not be sung after the Antiphon and Verse. In that case the Antiphon may be repeated after every Verse or two Verses. When the priest reaches the altar, the Psalm is if necessary interrupted at the end of a Verse, {\it Gloria Patri} is sung, and finally the Antiphon.

II. \quad When the Antiphon is over, the choir sings the {\it Kyrie eleison} thrice, the {\it Christe eleison} thrice, and again the {\it Kyrie eleison} thrice, alternately with the cantors, or with the other half of the choir. But the last {\it Kyrie eleison} is divided into two or three parts, marked by a single or double asterisk. It there be only two parts, and hence only a single asterisk, the first part is sung by the cantors or by the first half of the choir, the second part by the full choir. If there are three parts, the first being marked by the simple asterisk, and the second by the double one, then, the first part is sung by the other half of the choir: and the third part is sung by both sides together. Sometimes there are even five parts" then the manner of dividing the alternations in the chanting is marked by the single or double dividing the alterations in the chanting is marked by the single or double dividing sign being several times inserted; what has been said above sufficiently explains the execution.

III. \quad The priest alone in a clear voice gives the Intonation of the {\it Gloria in excelsis Deo,} and then {\it Et in terra pax hominibus,} etc., is continued by the choir divided into two parts, which answer each other, or else the full choir sings in alteration with the precentors. Then follows the response of the choir to the {\it Dominus vobiscum.}

IV. \quad After the Epistle or Lesson one or two cantors give the Intonation of hte Responsory, which is called the Gradual, as far as the sign *, and all, or at any rate the cantors chosen, conclude the chant with due care. Two sing the Verse of the Gradual, and, after the final asterisk, the full choir finishes it; or else, if the responsorial method is preferred, the full choir repeats the first part of the Responsory after the Verse is finished by the cantors or cantor.

If {\it Alleluia, Alleluia} is to be said with the Verse, the first {\it Alleluia} is sung by one or two voices as far as the asterisk * : and then the choir repeats the {\it Alleluia} continuing with the neum or jubilus which prolongs the syllable {\it a.} The cantors next sing the Verse, which is finished by the full choir, as before, beginning at the asterisk. When the Verse is finished, the cantor or cantors repeat the {\it Alleluia,} and the full choir sings only the closing neum.

After Septuagesima, the {\it Alleluia} and the following Verse are left out, and the Tract is sung, its Versicles being chanted alternately by the two sides of the choir answering each other, or else by the cantors and the full choir.

In Paschal Time, the Gradual is omitted and in its place the {\it Alleluia, Alleluis} is sung with its Verse as above. Then one {\it Alleluia} immediately follows, which must be begun by one or two cantors until the neum is reached, when it is not repeated, but finished by the full choir. The Verse and one {\it Alleluia} are sung at the end, in the manner above described.

The Sequences are sung alternately, either by the cantors and the choir, and or else by the alternate sides of the choir.

V. \quad When the Gospel is finished, the priest fives the Intonation of the {\it Credo} (if it is to be sung), the choir continuing with the {\it Patrem omnipotentem,} the rest, according to custom, being sung either in full choir or alternately.

VI. \quad The Offertory is begun by one, two or four cantors, in the same way as the Introit, and is finished by the full choir.

After the Offertory Antiphon the choir may sing to the ancient Gregorian chants those Verses which it was once customary to sing at this place.

If the Offertory Antiphon is taken from a Psalm, other Verses of the same Psalm may be sung. In that case the Antiphon may be repeated after ever Verse or two Verses. When the Offertory is over, the Psalm ends with {\it Gloria Patri,} ad the Antiphon is repeated.

If the Antiphon is not taken from a Psalm, some Psalm suitable to the feast may be chosen. After the Offertory Antiphon some other Latin piece may be sung suitable for this part of the Mass; which, however, must not be prolonged after the Secret.

VII. \quad When the Preface is finished, the choir goes on with {\it Sanctus} and {\it Benedictus.} If these are sung to Gregorian chant they must be given without a break; if not, {\it Benedictus} may follow the Consecration. During the Consecration all singing must cease, and (even if there is a custom to the contrary) the organ or other instrument is silent. It is preferable that there should be silence from the Consecration until {\it Pater noster.}

VIII. \quad After the Response at the {\it Pax Domini,} the {\it Agnus Dei} is sung thrice: either by the full choir, the Intonation being given by one, two or four cantors each time: or alternately, hut in such a way as to have the {\it Dona nobis pacem,} or the word {\it sempiternam} in the Mass of the Dead, sung by the full choir.

After the Communion, the full choir sings the Antiphon which is thus named, the Intonation being sung by one, two or four cantors as in the case of the Introit.

The Communion Antiphon is sung while the priest is consuming the Blessed Sacrament. When there are other communicants, the Antiphon is begun when the priest distributes Communion. If the Antiphon is taken from a Psalm, other Verses of the same Psalm may be sung. In that case the Antiphon may be repeated after every Verse or two Verses; and when the Communion is ended {\it Gloria Patri} followed by the Antiphon is sung.

If the Antiphon is not taken from a Psalm, some Psalm suitable to the feast and to this part of the Mass may be chosen.

After the Communion Antiphon, especially if the Communion takes a long time, some other Latin piece suitable to the occasion may be sung.

IX. \quad The priest of the deacon sings the {\it Ite Missa est,} or the {\it Benedicamus Domino,} and the choir answers with the {\it Deo gratias} in the same tone.

In the Mass of the Dead, the choir answers {\it Amen} to the {\it Requiescant in pace.}

X. \quad It is possible that for a good reason some piece assigned to cantors or choir cannot be sung as noted in the liturgical books; for instance, the singers are too few, or not sufficiently skilful, or the chant or the rite is too long. In that case the only alternative allowed is that the whole piece should be recited {\it recto tono} (on one note), or sung to a Psalm tone; this may be accompanied by an organ.
