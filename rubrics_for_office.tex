II. RUBRICS FOR THE OFFICE

a) {\it General Rules.}

The canonical Hours of the Office are: Matins, Lauds, Prime, Terce, Sext, None, Vespers and Compline.

Among these, Matins, Lauds and Vespers are called the {\it greater Hours;} Prime, Terce, Sext, None and Compline, {\it lesser Hours.} Compline, however, is usually comsidered separately in the rbrics.

The divine Office is said wither {\it in choir,} or {\it in common,} or {\it alone.}

It is said {\it in choir} by a community that the Church's laws bind to the Office in choir; {\it in common,} by a community not bound to choir.

The following rules apply both to the Office {\it in choir} and {\it in common} (even if it is recited by only two or three persons), and also to recitation {\it alone} unless the contrary is expressly noted.

b) {\it The time when the canonical Hours should be said.}

The canonical Hours of the divine Office were composed for the sanctification of the carious times of the natural day. It follows that in order really to sanctify the day and to recite them with spiritual profit it is well to try to recite them at a time as close as possible to the true time of each canonical Hour.

{\it Matins,} for a good reason, may be said after mid-day of the preceding day, but not before two o'clock.

{\it Lauds,} the morning prayer, {\it in choir} or {\it in common,} are said in the morning; this is praiseworthy even in recitation alone.

{\it Vespers,} even in Lent and Passiontide, {\it in choir} or {\it in common,} are said after mid-day; which it is well to observe also in recitation alone.

As regards {\it Compline,} it is very fitting that alll who are bound to recite the divine Office, religious especially, should say it as final night-prayers, even if, for a good reason, Matins of the following day has already been said.

When Compline is thus said at night, the {\it Pater noster} after the $\Vbar.$ {Adjutórium nóstrum} is omitted, and in its place, {\it in choir} or {\it in common,} a reasonable time is spent in examination of conscience; then {\it Confíteor} and what follows is said in the usual way. It is well to observe the same rule in the recitation alone.

c) {\it Arrangement of the divine Office.}

Feasts which have no I Vespers, and which for any reason acquire them under the new rubrics, borrow everything from II Vespers, with the exception of anything that may be given as proper to I Vespers.

THE VARIOUS PARTS OF THE OFFICE.

a) {\it Beginning and end of the Hours.}

The canonical Hours, whether {\it in choir, in common} or {\it alone} begin as follows:
{\it Matins,} with the \Vbar. {\it Dómine, lábia méa apéries;}
{\it Lauds,} the {\it lesser Hours} and {\it Vespers,} with the \Vbar. {\it Déus, in adiutórium méum inténde;}
{\it Compline,} with the \Vbar. {\it Iúbe, dómne, benedícere.}

The canonical Hours, {\it in choir, in common,} or {\it alone,} end as follows:
{\it Matins} (if Lauds do not follow immediately), {\it Lauds, Terce, Sext, None} and {\it Vespers,} with the \Vbar. {\it Fidélium ánimae;}
{\it Prime,} with the blessing {\it Dóminus nos benedícat;}
{\it Compline,} with the blessing {\it Benedícat et custódiat.}

b) Conclusion of the Office.

The daily course of the divine Office concludes, after Compline, with the Antiphon of Our Lady with its V. and prayer, then the V. {\it Divinum auxílium;} except on the last three days of Holy Week and in the Office of the Dead.

The remission of faults and indulgences granted for the recitation of the prayer {\it Sacrosánctae} is attached to this final Antiphon of Our Lady.

c) {\it Hymns.}

The hymns are said at each Hour, as fiven in the body of this book. But they are omitted: from Matins of Mayndy Thursday until None of the Saturday in Easter Week, and in the Office of the Dead.

At the lesser Hours and at Compline the hymns assigned to those Hours are always said, except at Terce on Whit Sunday and during its Octave.

The proper hymns assigned to certain Hours are never transferred to another Hour.

Unless the contraty is expressed in these rubrics, each hymn is always said with the conclusion found in the {\it Liber Usualis,} excluding any change on account of a feast or seasson.

Commemoration of another Office never involves a proper doxology at the end of the hymns of the Office of the day.

d) {\it Antiphons.}

The antiphons are said at all the Hours before and after the psalms and canticles; one or more, according to the Office and the Hours, as shown in their place. But they are omitted at the lesser Hours and at Compline: on the last three days of Holy Week, on Easter Day and during its Octave, and at the Office of the Dead on November 2.

The anthiphons are anways said {\it entire,} before as well as after the psalms and canticles, at all the Hours, both greater and lesser. The asterisk after the opening words of the antiphons shows how far the intonation continues.

If the proper antiphons assigned to certain Hours cannot be said, they are omitted, not transferred.

In paschal time, {\it allelúia} is added at the end of the antiphons if there is none already. But from Septuagesima until Easter Eve, {\it allelúia} at the and of any antiphon is omitted.

e) {\it Psalms and canticles.}

When a psalm or canticle begins with the same words as those of the antiphon, these words are not repeated, and the psalm or canticle is begun with the word that follows those of the antiphon; provided {\it allelúia} is not to be added to the antiphon.

A psalm that cannot be said at the Hour wo which it is secially assigned is not said at another Hour, but omitted.

At the end of the psalms and canticles, except the canticle {\it Benedícite, Glória Pátri} is said; but it is omitted on the last three days of Holy Week.

In the Office of the Dead, however, instead of the V. {\it Glória Pátri,} the {\it V. Réquiem aetérnam} is said, as shown in its place.

The Athanasian Creed is said only on Trinity Sunday, at Prime, after the psalms and before the repetition of the antiphon.



