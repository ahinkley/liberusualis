\section{RULES FOR INTERPRETATION.}

There are two notations in actual use in Plainsong: the square traditional notation and its modern transcription on the five-line stave; we put them side by side.

Musical notation, to be practical, must represent both melody and rhythm. Melodic signs or notes represent the relative pitch of sounds; rhythic sighs, the length of sounds and the rhythmic movement of the melody. The only notes used in Plainsong are those of the Diatonic Scale of Dob with the sole addition of the flat.

\subsection{I. The Stave, the Chef, the Guide, the Flat.}

The Stave.

In the traditional notation the Stave is of four lines only; lines and spaces were counted upwards thus:

XXX

When, as in figured music, the notes go beyond the stave, small lines (leger lines) are added, but never more than one, above or below the stave.

The Clef.

The Clef written at the beginning of the stave gives the name and place of the notes on the stave. There are two clefs in use: the {\it Doh} Clef: XXX marking the place of the {\it Doh}; the {\it Fah} Clef: XXX marking the place of the {\it Fah}.

These are simply the archaic forms of C and F. The {\it Doh} clef is placed sometimes on hte second line, often on the third line and very often on the fourth line. The {\it Fah} Clef is placed, as a rule, on the third line, rarely on the fourth line (e.g. offert. {\it Veritas,} pp 000).

This shifting of the Clef is merely to enable melodies of different range to be written on the stave. Once the note indicated by the Clef is known, the reading of the other notes is only a matter of practice. For the fluent reading of Plainsong and the transposition of the melody at sight, the Tonic Sol-fa system is invaluable -- almost indispensable -- for the ordinary reader.

The Guide.

At the end of each stave line is a sign called the Guide indicating in advance the first note on the following stave. This sign is also used in the course of the same line when the extended range of the melody requires for its writing on the stave a change in the place of the Clef. Here the guide shows, in relation to the previous notation, the actual pitch of the first note after the change.

XXX

In this example the {\it Doh} following hte change is sung at the same pitch as the {\it Doh} of the Guide; there is a change of Clef only, not a change of pitch. See Antiph. {\it Cum appropinquaret,} p. 000, in which a change in the position of the Clef occurs three times.

The Flat.

In the Vatican Edition the Flat holds good: {\it a)} for a whole word; {\it b)} as far as the first bar line -- even quarter-bar -- which occurs after it. The Ta returns to its natural state with a new word, after any bar-line and, of course, whenever a natural XXX is placed bevore it. In a few pieces the Flat is placed near the Clef on each section of the stave; it then makes flat every Te or B in the piece unless contradicted by a natural.

\subsection{II. -- The Modes, the Choice of Pitch, Bar-Lines.}

For the benefit of those accustomed to modern music only, and in order to avoid any misunderstanding in the reading of Plainsong, a few remarks may here be mae on the Modes, the choice of Pitch, and Bar-lines.

The Modes.

In modern music there are only two Modes or Scales in general use: the Major Scale built upon {\it Doh}, and the Minor Scale built upon {\it Lah} as the key note. The various keys in which these two Scales can be played or sung, aggect only the pitch of the notes; they remain the same Scales, only at a different pitch. A cursory glance at the Plainsong melodies, whether in the old or in the modern notation, might easily give the impression that they are in the Scale of {\it Doh}. Indeed they are read and treated as such by the ordinary singer. But this is only an optical or auricular illusion which a further glance at the tinic, or the characteristic intervals would dispel. While it is true to say that the melodies use only the material of the diatonic Scale of {\it Doh} (with the important exception of XXX), we must not therefore conclude that they are necessarily or even frequently in the Scale or Mode of {\it Doh}. The numerical placed at the beginning of each piece would tell us otherwise. Apart altogether from the question of transposition, each note in this simple material of the Scale of {\it Doh}, can become in its turn a modal tonic, provisional or final, regardless of where the semitone falls. Hence -- if we also mention the difference of range and the modulations -- a variety in mode or scale of which even the medieval theory or eight Modes gives us an imperfect idea. In this respect the resourcefulness and variety of Plainsong far outstrip modern music. The following example in which each little formula makes us hear a different modal tonic and gives the impression of a different Mode or Scale, illustrates, with the simple material of the Scale of {\it Doh}, this richness and variety:

XXX

The Pitch.

It must be clearly understood that in Plainsong the notation is not and was never meant to indicate the absolute but only hthe relative pitch of the melodic intervals. The two Clefs of {\it Doh} and {\it Fah}, and their different positions, have no other aim than to make possible or easier the writing of the melodies on the stave. Often these clefs could be interchanged, their positions could be different, as, in fact, is the case in the Manuscripts of Plainsong. It must, therefore, be well understood that the notes read on the stave are to be sung at the pitch which is within the compass of the singers, according to the size of the building, and the special character of the piece.

Bar-Lines.

It will be noticed that the bar-lines of modern music do not occur in Plainsong. This does not mean that there is no time or measure, but that there is no time or measure in the modern sense, and that there is no "strong beat" or 
accent" occuring at regular intervals. Plainsong is an entirely different idiom. Its time like its rhythm is free -- a free interlacing of binary and ternayr groups (of course at the discrection not of the singers but of the composer) which, like the prose text which they clothe, glide along freely, in order and veriety, forming periods with sections and phrases of unequal length and importance.

\subsection{III. --- The name, shape and value of notes, and the names of neums or groups.}

A. -- {\sc Single notes.} -- with our without rhythmic signs.

Two kinds of notes only are used as single notes i.e. notes standing alone;

1. The square punctum
2. The Virga
The Modern transcription of these

Explanation:
{\it Column 1.} In this column, the single notes without rhythmic signs have the value of a quaver in modern music. And as in modern music we usually have two or three quavers to the beat, so likewise in Plainsong, we have two or three single notes forming a compound beat or rhythmic group.

Three rhythmic signs may be added to these single notes; hence in
{\it Column 2.} The vertical episema, (a) which marks the beginning of a compound beat and the rhythmic step of a movement, (see further on XXX)
{\it Column 3.} The horizontal episema which indicates a slight lengthening of the note. This stroke may also cover a whole group, but in such cases, the lengthening must not be too marked, in order to aintain the rhythmic unity of the group.

As regards the value of the lengthening, a good rule is: the oftener it occurs, the less we must mark it and vice versa. Notice also that, while the note lengthened by the horizontal episema may often be strong from its position in the melody or text, more frequently perhaps it must be weak; intensity is not inherent in any rhythmic sign.

{\it Column 4.} Here both the vertical and the horizontal episema are attached to the same note, thus indicating at once the beginning of a rhythmic group and a slight lengthening during which the voice dies away in order to mark the conclusion of a small melodic phrase.

{\it Column 5.} Here the dot doubles the note which precedes it.

B. -- {\sc Neums or groups of two notes.}
1. Ascending:
Podatus
2. Descending:
Clivis
3. On the same degree:
Bivirga
Distropha

C. - Neums or groups of three notes.
1. Ascending
a) Scandicus
b)
1st form: always with a vertical episema under the 2nd note:
2nd form: with the two first notes on the same degree: (a)
2. Descending:
Climacus
3. With the 2nd note of the group higher than the two others:
Torculus
Lower than the two others:
Porrectus
4. On the same degree:
Tristropha

D. -- Compound neums or groups of more than three notes.
Pes or Podadus sub-bipunctis
Porrectus flexus
Torculus resupinus
Salicus flexus
Scandicus flexus
Climacus resupinus


E. -- Special neums or groups.

1. Liquescent neums or groups. (a) \footnote{We mention the strange names of these groups for the sake of completeness; there is nothing otherwise mysterious about them. They are sung in the ordinary way (See further, Liquescent Notes).}

Epiphonus or liquescent Podatus
Liquescent Porrectus
Liquescent Torculus
Liquescent Scandicus
Cephalicus or liquescent clivis
Ancus or liquescent Climacus

2. Oriscus.

3. Pressus.

4. Quilisma.

\subsection{IV. -- Remarks on come of the above notes or groups.}

1. Each note in Plainsong, whether isolated or in a group, whatever be its shape, has the same value, the value of a quaver in figured music; followed by a dot, its value is equivalent to a crotchet. Evenness and regularity of the notes is the first and essential condition of a good rendering of the chant. In syllabic chant no sullable or note must break this refularity, yet here especially the light, uplifted accent of the words must give life, variety and movement to the singing. In neumatic pieces (those with groups) care must be taken to keep the exact relative value of the dimple, double and triple notes. Very frequently a single noteimmediately preceding a long note and, still more, a single note between two long ones, is not given its full value; the same fault occurs in the case of the last note of a group leading to another group.

2. The {\it virga} is sometimes repeated; it is then called a {\it bivirga} or double virga which is represented in modern notation by a crochet or two tied quavers. For example:
XXX

3. The {\it apostropha} is never used alone; it may occur twice (distropha) or thrice (tristropha), and these again may be repeated:
XXX

The Tristropha is frequently met with in this form:
XXX

Formerly each of these two or three notes was characterised by a slight stress or impulse of the voice; in practice, we advise the joining of the notes in one sound. These double or triple notes, especially when repeated, may ne sung with a slight crescendo oor decrescendo according to their position in the word of the text or in the melodic line. A gentle and delicate repercussion (i.e. a fresh layer of sound) is needed at the beginning of each distropha or tristropha (cf. next example A. B. C. D.), as well as on the first note of any group which begins on the same degree as the strophicus. (Cf.  examples E. F.). In the following examples the place of this repercussion is marked by the vertical episema, but usually the rule will be known.

XXX

4. {\it The podatus:} two notes, the lower of which is sung first; the higher note should be sung gently, and gracefully rounded off. If the {\it podatus} forms part of the upward movement of the melody and comes on the accented syllable of a word, its first note must receive a marked impulse.

5. {\it Scandicus and climacus:} these groups may be made up of three, four, five, or more notes. Care must be taken to have regularity in these froups, especially not to slide the descending diamond notes of the Climacus, which, notwithstanding their shape, have exactly the same value as the Virga at the beginning of the group.

6. {\it The salicus:} this group, as a rule of three notes, may also be made up of four or five notes. Not to be confused with the Scandicus, it can be recognised by the vertical episema placed {\it under} one of the notes which make up the group. The note thus marked should be emphasised and lengthened, just as in the case of the note which precedes the quilisma. (see 13):
XXX

If in an ascending froup the vertical épisema is placed {\it over} one of the notes it indicates a rather important ictus which should be brought out in the rendering. In the case of a group giving {\it an interval of a fifth,} the upper note whenever marked with the vertical episema should be notably lengthened:
XXX

7. {\it The torculus:} three notes, perfectly equal in length, the top one gently rounded off.

8. {\it The porrectus:} three notes, the first two of which are placed at the extremities of the thick oblique stroke:
XXX

9. {\it The flexus, resupinus, subbipunctus, subtripunctus:} for practical purposes these need not be studied; they are technical traditional names for compound groups which are rhythmically divided and sung according to the ordinary rules. (see further on XXX).

10. {\it Liquescent notes:} these are printed in smaller type but this does not affect their duration nor their execution, except in that they must be sung lightly. They occur when two vowels form a diphthong ({\it au}tem, {\it eu}ge), or at the junction of certain consonants (Hosa{\it nn}a, to{\it ll}is, mu{\it nd}i, etc.) or to introduce the semiconsonants {\it j} or {\it i} (e{\it j}us, allelu{\it i}a).

11. {\it Oriscus:} a note placed at the end of a group, on the same degree, and very often at the end of a Torculus (see table of Neums); it forms a double note with the preceding one and should be treated lightly.

12. {\it Pressus:} two notes placed side by side on the same degree, the second of which is first of a group. This may occur in two ways:
a) by a {\it punctum} being placed just before the first note of a group:
XXX
In the MSS some of these groups are not a {\it Pressus}, but Solesmes treats them as such in view of the notation of the Vatican Edition.

b) By the juxtaposition of the two neums, the last note of the first being on the same degree as the first note of the second.
XXX

Podatus and Clivis XXX
Climacus and Clivis XXX
Clivis and Clivis XXX
Scandicus and Climacus XXX

The two notes placed side by side in the Gregorian notation combine when sung to form one note of double length; in the {\it pressus}, the ictus is placed on the first of these two notes.

13. {\it The quilisma}: this jagged note XXX is always preceded and followed by one or several notes (see examples in the table of neums); its value is the same as that of other notes, but it must always be rendered lightly. The note immediately {\it before} the quilisma should be notably lengthened, and the most emplasised of the whole group even when preceded by a double note.

\subsection{V. -- Pauses, Breathing.}

A single note has exactly the same value, in intensity and duration, as the syllable to which it is united. The approximate value of a syllable may be reckoned as a quaver.

Like speech, a Plainsong melody may be divided into periods, sectionsm and phrases, in which the last note or the last two notes of each division are lengthened. And as in speech, so here also, pauses must be in proportion to the importance of the rhythmic divisions. The ending of each division should always be softened; if a division ends with two double notes, the last should be the more notably softened, and no fresh impulse of thevoice should be noticed on it.

1. The endings of short unimportant phrases do not, as a rule, allow the taking of breath; these are marked by the two episema attached to a punctum (square XXX or diamond XXX), or more rarely by a dotted note (XXX), sometimes followed by a quarter XXX or half bar. XXX

2. More important groups, forming small sections, are indicated in the same way; these are always followed by a quarter or half bar.

Often however, these quarter or half bars, especially in the shorter antiphons, merely indicate the rhythmic subdivision, and no breath should be taken.

3. The end of a section properly so called, which is made up as a rule of two or more phrases, is indicated by a half bar. Here it is generally necessary take breath, yet without break in the movement; hence the dotted note before the bar must necessarily be shortened slightly -- almost to half its value.

Example:
XXX
which must be rendered in this way:
XXX

4. Finally the close of a whole period is marked wither by a full bar, or by a double bar at the end of the piece, or at the end of an important division of the piece. Here breath must be taken and a longer pause be made. In the modern notation this pause is indivated by a quaver rest which is placed before or after the bar line, in accordance with the rhythm of the phrase following.

In pieces normally sung by alternating choirs (as in the {\it Kyrie, Gloria, Credo} etc), the double bar indicates a change of Choir. In such cases the pause will vary according to its importance in the melody and the text.

5. {\it The Comma} allows only a rapid breath without break in the movement, hence during the value of the preceding note:
XXX

6. Though breath need not necessarily be taken at every bar-line, yet, on the other hand, in the more elaborate pieces (graduals, alleluia, offertories) it is sometimes necessary and indeed excellent to breathe between the bars marked in the book. It is impossible to give minute rules for doing this correctly. All that need be said here is that breath must be taken in such cases {\it a)} without interrupting the rhythmic movement and regularity, or changing the value of the notes, {\it b)} according to the melodic phrasing, hence at the end of small melodic groups, {\it c)} at a long note, if possible, in order to allow more time for breathing.

7. The end of the Intonation and the entry of the choir are marked by a star in the verbal text. A dotted note or horizontal episema indicates the rhythmic punctuation suited to each case; sometimes the melodic sense admits of no pause; then all punctuation marks are omitted.

\subsection{VI. -- Notes on rhythm, the vertical episema, rhythmic step or alighting point.}

Rhythm in singing is a movement of the voice wherein it successively rises and falls. It is in the well-ordered succession of such movements that rhythm essentially consists. In its elementary form, the rise or {\it arsis} is the beginning of a rhythmic unit or movement; the fall or {\it thesis} its end. The rhythmic fall or {\it thesis} will necessarily occur on every second or third note in the course of the melody -- like the fall in every second or third syllable or the words which accompany it. Hence the impossibility of two such falls occuring in immediate succession, unless, of course, the first be a note of double calue. But notice carefully that these steps or falls form in an ascending movement the arsic part, or rise, of the larger rhythm, just as every step one takes in climbing up a hill goes to the general movement upward. This whole movement upward is known as the arsic part of the larger rhythm. Similarly when the movement is downward, every rhythmic rise or arsis of the voice forms part of the descent of the larger rhythm, just as in walking down a hill the regularly uplifted foot is part of the downward movement. This whole movement downward is known as the thetic part of the larger rhythm.

For the proper execution of Plainsong it is therefore nevessary to be able to recognise the place or each thythmic step, ictus, or alighting point, in order to secure order, regularity, and life. The following rules indicate the notes of the melody which must receive the rhythmic ictus:

Rule 1. All notes isolated, or in a group, which are marked with the vertical episema.

In figured music the compound beats or simply the beats (usually binary or ternary) are made clear in the notation, either by the grouping of the notes, or by the regularity of the time chosen and marked at the beginning of the piece.

Thus:XXX

Very often both these means are employed at the same time. When, however, we have no indication of time, no time-bars as in Plainsong, and no groups nor long notes, we shall be obliged to mark the beginning of the beat, ictus, rhythmic step, or alighting point, each time the notation does not mark it for us. This is the r\^ole of the vertical episema. And just as in figured music, certain beats are strong, others weak, others weaker still, so in Plainsong, the ictus or rhythmic step will be strong or weak according to its position in the melody and text.

Rule. 2. All sustained notes: {\it distropha, tristropha, bivirga, pressus, oriscus, dotted notes, and the note before a quilisma.}

It should be noticed that although a note lengthened by an {\it horizontal episema} generally receives the ictus or rhythmic step, this need not always the case. (Example: the word {\it corda} in the Alleluia: {\it Veni Sancte.} p. 000)

Rule. 3. Any note which begins a group.

Example: XXX

If we have to deal with composite neums, it is generally easy to resolve them into the simple groups of two or three notes of which they are composed. Notice only that, in this case, the Virga should be considered as the beginning of a new group:
XXX

This third rule holds good only when it does not clas with rules 1 or 2.

a) Example in which rule 1 takes precedence:
XXX

In a salicus of three notes, the note marked with the {\it vertical episema} must also be lengthened as though it were marked with a {\it horizontal episema} (Cf. above: Salicus). Were it not for the difficulty of writing it, the latter would have been used instead.

b) Another example in which rule 2 prevails is the case of teh {\it Pressus} or the {\it Oriscus.} Here the first of the two notes forming the double note is the place of the ictus or rhythmic step. Elsewhere it would be on the first note of the group.
Example: XXX
{\it and not:}
XXX
In this connection it should be noticed that the following: XXX is not a Pressus preceded by a punctum: XXX but a Distropha followed by a Clivis: XXX

As regards this last example it should be remembered that a repercussion (or fresh layer of voice) is required each time a note affected by the ictus is of the same degree as the one immediately preceding it.

Briefly then, the ictus or rhythmic step placed on the beginning of each group is disloged by a Pressus, or Oriscus, or by the vertical episema already marked in the text.

If, in applying the three rules given above, we meet with some passages containing more than three single notes from one ictus to the next, we shall have to put in between, as a stepping stone, an ictus of subdivision. If we have four notes, this of course will give 2 + 2; if we have five notes we shall divide them either 2 + 3 or 3 + 2, according to what seems to be suggested by the melody or text, and to be the more natural arrangement.

In syllabic passages in which there is no vertical episema, and no long note or group appears, we shall decide for ourselves in one or other of the following ways:

1. Either by counting back two by two, starting from the last certain ictus of each section:
XXX

2. Or by following the melody, and preferring, first the endings of the words, secondly the accented syllables, while avoiding as much a possible the weak penultimate syllables. This is often the more excellent way for those who are musically alert.
XXX

The acceptance of these principles governing the ictus does not necessarily imply agreement with their application in every instance. For the sake of uniformity, however, it is advisable to adhere to the current rhythmic grouping.

IMPORTANT NOTE: - As we have already said, the dynamic value or strength of the ictus or rhythmic step caries considerably. Sometimes it is strong, sometimes weak; everything depends on the syllable to which it corresponds and the position it occupies in the melody \footnote{(a)}{It is well known that from the text point of view the syllable or sellables after the accent must be relatively weak, while from the melodic point of view the great rule is: a slight and gentle crescendo in the ascending, and a similar decrescendo in the descending parts. This must always be done without sharp contrasts or exaggeration of any kind.}. The fact therefore that this intensity varies is a proof that the ictus belongs not to the dynamic but to the rhythmic order; its being and influence are contributed and felt by elements from the melody and the text. The expression "the ictus is more in the mind than in the voice". has sometimes been misunderstood. The meaning will, perhaps, be clearer if we say that it is felt and intimated by tone of voice rather than expressed by any material emphasis. When in addition to the independence of rhythm and intensity, we consider that the Latin accent is light, lifted up and rounded off like an arch, is not heavy or strongly stressed, is arsic and not thetic, we shall not be surprised to meet frequently in Plainsong accented syllables outside and independent of the ictus or rhythmic step. \footnote{(a)}{The light and arsic character so essential to the Latin accent must always be brought out even when it coincides with the rhythmic ictus. When, as often happens, a single note is put on the accented syllable and a number of notes is put on the weak penultimate syllable, it is very important to round off and bring out gently the arsic character of the accent, v. g. XXX} Indeed the Plainsong masterpieces of the golden age clearly assert this independence. And this is perpectly musical, in full accord with the genius of the Latin language and the Roman pronunciation and accentuation so much desired by Pius X. To place the ictus or rhythmic step always and necessarily on the accented syllable, as modern musicians are wont to do in another idiom, would be, we maintain, to spoil the rhythm and melody, accent and words of our venerable melodies.

\subsection{VII. -- The basis of plainsong rhythm.}

We have already defined the rhythm of Plainsong as a movement of the voice wherein it rises and falls in orderly fashion. It is a free interlacing of binary and ternary groups of notes so well balanced as to convey to, and produce in the mind a sense of order in the midst of variety. We constantly meet with this order in variety in all forms of art, indeed in nature itself. It is the mind's delight. Rhythm of every kind moves stepwise, but not necessarily with fixed mechanical regularity. All that is essential to it is proportion, balanced movement and repose, rise and fall, the due correlation and interdepencence of parts producing a harmonious while. Such is free rhythm, the rhythm of Plainsong.

The Plainsong composers -- much less the interpreters -- did not create this rhythm; they found it in outline, already in existence, in the Latin prose text which their music is intended to clothe and adorn. We must never lose sight of the fact that Plainsong is vocal Latin music, for this is the key to the understanding of its rhythmic and melodic structure. It has been grafted on, and has sprung out of, the natural rhythm and melody of the Latin words, phrases, sections, and periods for which it has been written.

In the Latin word the accented syllable is the vital arsic element; the final and weak penultimate syllables are soft, relatively weak, and thetic. Thys there is movement and repose, rhythm of an elementary kind in every word. Words of two syllables often intermingle with those of three syllables, thereby giving variety and interest to the rhythm of the text. e.g.

XXX

Each word is in itself a small rhythm which ends with the endings of the words. A succession of these small rhythms creates the small measure, the time form one thesis, step or ictus, to the next; the group of notes thys created forms a compound beat. v.g.

XXX pp.xxxj.


Just as is 6-8 time e.g. three quavers form a beat, so in Plainsong the individual notes of the small measure -- the notes from one ictus to the next -- group themselves two by two or three by three and are treated like slurred notes in modern music. On the violin they would be played "in one bow". These small measures are again stitched into and form part of the larger grouping in the general design which must never be lost sight of:

XXX

Another example with words, in which we find $\frac{2}{4}$ time:

XXX

Hence the fall or thesis of each rhythm is the beginning of each little measure or each compound beat. The interlacing of words of two and three syllables determines the corresponding interlacing of binary and ternary measures or beats. If, as often happens, we have more than three syllables in a word, groups for example of four, five, six, or seven syllables, these as in music will be naturally be divided into the simple elements of two and three, keeping thereby the stepwise movement of the rhythm. e.g. 

XXX

There is yet another element of rhythm. In reading the text, we observe (though they are not given in the MSS) the various punctuation marks and pauses which are nevessary no only for the maning of the text but also fr the appreciation of the larger rhythm. The melody which is designed for the text is also divided into periods, sections, and phrases, each with its due pause and with its last note or notes lengthened and softened. These divisions, marked in all modern Editions (but not in the MSS) by the differnt bar lines, correspond to the natural phrasing of the text both musical and verbal, and are an indispensable condition of the wider rhythm. Again, in the verbal text there are further rhythmic subdivisions and groupings left unmarked in the text which nevertheless must be felt by the reader and intimated by tone of voice rather than by any material emphasis. Similarly, in the melodic text, there are rhythmic subdivisions and regroupings which are more difficult to recognise and define. These again must be felt by the singer and intimated in teh voice. In vertain cases they are marked by the vertical episema in the Solesmes Editions, but marked or unmarked they must be taken account of by everybody; they are an extension of the principle at work in the introduction of bar lines. They give a foothold, balance and cohesion to the rhythm, and are inplied in the natural rhythm of the words, or the rhythm of the melody, or the rhythmic indications of the manuscript.

Thus we see the principle which governs the rhythm of Plainsong. Once found in, and taken from, the Latin text, it has been applied instinctively by the Gregorian composer to the whole Gregorian art. But the comoser is an artist, not a mechanic; the verbal text is the take-off of his flight. The melodic order has often suggested or imposed a rhythmic grouping independent of the words taken by themselves. The composer's artistic genius, as we see in the manuscripts, has often stressed certain notes, and in this way suggested such and such a rhythmic grouping. Because of its connection with the melodic element, the verbal rhythm has, at the same time, developed into a musical rhythm with its own laws of tonality, modality and beauty, until, in the more ornate pieces, we have musical rhythm only. But this rhythm always keeps its freedom, a freedom determined on each occasion by the natural rhythm of the words, the actual elements of the melody or the indications of the Manuscripts.

\subsection{VIII. -- Rules for the Chanting of the Psalms.}

A Psalm-tone consists of the following parts: {\it a)} The Intonation. {\it b)} The Tenor, Dominand or Reciting note. {\it c)} The Cadences, the first of which is in the middle of the verse before the star, and is therefore called {\it the Mediation;} the second is at the end, and is therefore called the {\it Final Cadence.}

When the first part of the verse is very long it is subdivided by {\it a Flex} (marked by a cross $\dag$), so called because the voice bends down or drops to a lower note ){\it flectere,} to bend) which is doubled. Here, if nevessary, breath may be taken, yet without break in hte movement.

The simple and solemn formulae both for the {\it Eigh Tones} and the {\it Tonus Peregrinus} are fully set out in this book at the beginning of each Psalm.

{\it The Intonation} is a formula at the beginning of the Psalm which connects the Amtiphon with the Tenor or Dominant. It is made up of two or three notes or groups adapted to the syllables. Intonations of two notes or groups are adapted to the first two syllables of the verse; those of three notes or groups are adapted to the first three syllables. There is no exception to this rule.

In ordinary Psalmody the Intonation is used for the first verse only; the other verses begin directly on the Tenor or Reciting note. Whenever the Intonation has to be repeated for each verse -- as in the Magnificat -- this is always indicated.

When several Psalms or several divisions of a psalm (with {\it Gloria Patri} for each division) are chanted under the same Antiphon, the first verse of each should be intoned by the cantor as far as the Mediation. )Cf. Compline 000 et seq).

{\it The Tenor, Reciting note} or {\it Dominant} includes all the notes which are sung at the same pitch from the Intonation to the Mediation and from the Mediation to the Final Cadence. Here the rules of good reading and phrasing are important, avoiding dull monotony by the delicate relief given to the accents expecially in the more important words. There must be no cut or break interrupting the regular flow of the recitation from the beginning to the Mediation and thence to the Final Cadence. There must be life and movement but no hurry; the singing is the "Opus Dei" -- God's work.

{\it Cadences.} In this book the Cadences of each Tone or mode are set out at the beginning of the Psalm. The choice of the Final Cadence, where there are several, is determined by the Antiphon.

Cadences are of two kinds:

XXX

A. Cadences of one accent.

XXX

B. Cadences of two accents.

XXX

It will be noticed that when a word {\it accented on the third last syllable} occurs, an extra note, (printed hollow thus XXX), has to be used in the Cadence. Moreover, a great many Cadences have one, two or three syllables preparatory to the accent. In the Psalms of the Office these Cadences are easily recognised because they leave the Reciting note in a descending movement. \footnote{(a) Two exceptions may be mentioned: the Meditation Cadence of the Solemn 5th Mode, and that of the 'Tonus in directum" used at Compline on Holy Saturday.}

XXX

The passage from the Reciting Note to the accent is made by one, two or three preparatory syllables, e.g.

What has hitherto been said suggests four questions of practical interest; we mention them here with the solutions given with the {\it Psalms for Vesters} (000. et seq).

1. How determine the choice of Cadences suitable to the words?
Answer: by looking at the Rubric at the beginning of each Psalm.

2. Which syllables are to be fitted to the accented notes in each Cadence?
(Accent here includes not only the tonic but the secondary accent as also any syllable taking the place of the accent.)
Answer: those which are printed in heavy type.

3. Which Syllables should be fitted to the notes or groups of notes preparatory to the Accent?
Answer: those printed in Italics.

4. At what pitch should the extra note (printed hollow thus XXX) be sung?
Answer: its pitch is shown in the first verse of the Psalm.

\begin{center}
{\it An Extra note for the Accent when a Clivis occurs in the Cadence.}
\end{center}

With a word {\it accented on the second last syllable} e.g. Redémptor, no difficulty arises, for then there is no extra note required and the Clivis is sung on the accented syllable. But with a word {\it accented on the third last} syllable, the accent is not sung on the Clivis but on an extra note placed immediately before it. This is done to preserve the smoothness of the Cadence.

A. Accent on the {\it second} last syllable.

XXX

B. Accent on the {\it third} last syllable.

XXX

But, it may be asked, how are we to recognise such Cadences and know the pitch of the extra note before the Clivis? When necessary, a Rubric forwarns us of their presence in a Psam. They are also indicated on the stave by a bracket over the extra note and the Clivis (see example above), while in the text, the accent and the syllable sung on the Clivis are printed in heavy type.

{\bf Solen psalmody.}

The Solemn formulae for all the Tones are printed with the Magnificat (000-000); they are classified in the same way as the Simple Tones. The Vatican Edition regards a Solemn Cadence of the 1st and 6th Modes as a Cadence with {\it two accents.} \footnote{(a)}{This Cadence is not derived from the Simple Psalmody of the 1st Mode bt from a Simple VI Mode Cadence of {\it one accent} with three preparatory notes.}

As we have received permission from Rome to consider it {\it ad libitum} as a Cadence with {\it one accent} and three preparatory notes, we have availed ourselves of this; thus all Cadences of the same design (i.e. leaving the Reciting note in a descending movement) can be treated in the same way. \footnote{(b)}{When we received this permission, which all may use, it was pointed out that our method was in conformity with the Decree of July 1912 on the rendering of monosyllables and Hebrew words in the Lessons, Versicles and Psalms.

XXX

{\bf Tonus peregrinus.}

We have also permission to add a G {\it ad libitum} before XXX in the mediation of the {\it Tonus Peregrinus.} By this means the formulae becomes quite regular, leaves the Reciting note in a descending movement, and has one accent with three Preparatory syllables.

XXX

\subsection{IX. -- The readong and pronunciation of liturgical latin.}

Plainsong being vocal and Latin music, neither its rhythm nor its melody can be rightly appreciated or sung apart from the meaning of the text, the correct pronunciation of the words, and their proper grouping into phrases. In other words, there must be good diction. No Choir should attempt to sing a melody before reading the text correctly and fluently. Nor is a knowledge of music sufficient; one must somehow understand the Latin text and its liturgical content and cultivate a kindred spirit in order to interpret aright the accompanying melody.

For good diction we must also cultivate a rhythmic sense; verbal rhythm and accent are of first-rate importance. It must always be remembered that while the accented syllable is the vigorous, life-giving, arsic element in a Latin word, the final and weak penultimate syllables are always soft, relatively weak, and thetic. Thus there is movement and repose, rise and fall -- rhythm of an elementary kind -- in every Latin word.

XXX

Ordinarily, in a Latin sentence, words of two syllables freely interlace with those of three syllables, and form a larger rhythm which is the charm of the well-balanced Latin prose of our great classical Collects (see XXX).

XXX

In good Latin diction -- listen to a Roman Professor lecturing in Latin -- the tonic accent stands out clearly, is lifted up lightly, rounded off and slighly lengthened, yet as the time-value of a single, not a double note in music. Thus there is no flat, dull monotony which is indeed the {\it execution} (so gruesome to those listening) of many a venerable Plainsong Recitative. Good diction means good phrasing also, and the intelligent use of the Phraseological Accent. For just as the tonic accent gives cohesion and life to the word, so the phraseological accent draes together the separate words into groups, and gives a tonic prominence and influence to the important word, phrase, and pause. Thus the listeners are made to understand the text; they feel that the reader understands it also.

The correct pronunciation of Latin words, vowels and consonants is a rock of offence to many people. We are not here concerned with the delicate question of pronunciation in the Classical period, but only with the pronumciation of the living liturgical Latin of the Church. Our aim, in compliance with the wishes of his holiness Pius X, is to pronounce and speak Latin in the Roman Style so eminently suitable to Plainsong. For our purposes the vitally important element in this style is the rich, open, warm sounds of the vowels A and U. The other elements will, to be sure, receive our close attention; this one is primary and indispensable. Sing a piece of Plainsong, opening the mouth well, bringing out fully these vowel sounds; the effect is delightful, we realize immediately what a splendid difference thy make. We must be careful also to give every syllable its full calue, and not to slur over or clip off the weak penultimate syllable in a word. It is a common age-long fault (which has formed many words in French and Italian) to do this, and pronounce e.g. {\it Domine} as if it were {\it Domne} and {\it dextera} as if it were {\it dextra{. Very often, as if to prevent this, the early composers of Plainsong but a bunch of notes on such weak syllables, to the scandal of some moderns, who will confuse length with stress and accent.

Many have bever learned the Roman pronunciation or know it imperfectly. Besides its great importance in Plainsong it makes for that uniformity which inspired the Vatican Edition itself; {\it Unus Cultus, Unus Cantus.} We therefore give a list of the corect pronunciation of the vowels and consonants to which reference can be made in case of doubt; it is advisable to peruse it form time to time.

Vowels and Diphthongs.

Each vowel has one sound; a mixture or sequence of sounds would be fatal to good Latin pronunciation; this is far more important than their exact length.

It is of course difficult to find in English the exact equivalent of the Latin vowels. The examples given here will serve as an indication; the real values can best be learned by ear.

{\bf A} is pronounced as in the word {\it Father}, never as in the word {\it can}. We must be careful to get this open, warm sound, especially when {\it A} is followed by {\it M} or {\it N} as in {\it Sanctus, Nam,} etc.

{\bf E} is pronounced as in {\it Red, men, met;} never with the suspicion of a second sound as in {\it Ray}.

{\bf I is pronounced as ee in {\it Feet}, never as i in {\it milk} or {\it tin}.

{\bf O} is pronounced as in {\it For}, never as in {\it go}.

{\bf U} is pronounced as oo in {\it Moon}, never as u in {\it custom}.

{\bf Y} is pronounced and treated as the Latin {\bf I}.

The pronunciation given for {\bf i}, {\bf o}, {\bf u}, gives the approximate {\it quality} of the sounds, which may be long or short; care must be taken to bring out the accent of the word.

e. g. {\it mártyr = márteer}.

As a general rule when two vowels come to gether each keeps its own sound and constitutes a separate syllable.

e. g. {\it diéi} is {\it di-é-i; fílii} is {\it fí-li-i; eórum} is {\bf e}{\it -ó-rum}.

THis applies to {\bf OU} and {\bf AI :}
e. g. {\it caelum}.

In {\bf Au,} {\bf Eu,} {\bf Ay} the two vowels form one syllable but both vowels must be distinctly heard. The principle emplasis and interest belongs to the first which must be sounded purely. If on such a syllable several notes are sung, the vocalisation is entirely on the first vowel, the second being heard only on the last note at the moment of passing to the following syllable.

Examples : {\it Lauda, Euge, Raymundus.}

XXX

{\bf EI} is similarly treated only when it occurs in the interjection:

{\it Hei = Hei,} otherwise {\it Méi = Méi,} etc.

{\bf U} preceded by {\bf Q} or {\bf NG} and followed by another vowel as in words like {\it qui} and {\it sanguis,} keeps its normal sound and is uttered as one syllable with the vowel which follows: {\it qui, quae, quod, quam, sanguis.} But notice that {\it cui} forms two syllables, and is pronounced as {\it koo-ee.} In certain Hymns, on account of the metre, this word has to be treated as one syllable (Cf. {\it Major Bethlem cui contigit.} Lauds for the Epiphany).

{\bf Consonants.}

The consonants must be articulated with a certain crispness; otherwise the reading becomes unintelligible, weak and nerveless.

{\bf C} coming before {\bf e, ae, oe, i, y} is pronounced like {\it ch} in {\it Church.}
e.g. {\it caelum} = {\it che-loom; Celília = che-cheé-lee-a.}

{\bf CC} before the same vowels is pronounced {\it T-ch.}
e.g. {\it ecce = et-che; cíccitas = seét-chee-tas.}

{\bf SC} before the same vowels is pronounced like {\it Sh} in {\it shed.}
e.g. {\it Descéndit = de-shén-deet.}

Except for these cases {\it C} is always pronounced like the English {\it K.}
e.g. {\it cáritas = káh-ree-tas.}

{\bf CH} is always like {\it K} (even before {\bf E} or {\bf I}).
e.g. {\it Cham = Kam, máchina = má-kee-na.}

{\bf G} before {\bf e, ae, oe, i, y,} is soft as in {\it generous.}
e.g. {\it mági, génitor, Regína.}

Otherwise {\bf G} is hard as in {\it Government.}
e.g. {\it Gubernátor, Vigor, Ego.}

{\bf GN} has the softened sound given to these letters in French and Italian.
e.g. {\it agneau, Signor, Monsignor.}

The nearest English equivalent would be N followed by y.
e.g. Ah-nyoh, Regnum == Reh-nyoom; Magníficat = Mah-nyeé-fee-caht.

{\bf H} is pronounced {\bf K} in {\it nihil (nee-keel)} and {\it mihi, (mee-kee)} and their compounds. In ancient books these words are often written {\it nichil} and {\it michi}. In all other cases {\bf H} is mute.

{\bf J} often written as {\bf I}, is treated as {\bf Y,} forming one sound with the following vowel.
{\it Jam = yam; alleluia = allelóoya; major = ma-yor.}

{\bf R}: when with another consonant, care must be taken not to omit this sound. It must be slightly rolled on the tongue v.g. {\it Carnis.}

Care must be taken not to modify the quality of the vowel in the syllable preceding the {\it R:}
e.g. Kýrie: Do not say {\it Kear-ee-e} but K'ee-ree-e
Sápere: Do not say {\it Sah-per-e} but Sáh-pe-re
Dilígere: Do not say {\it Dee-lee-ger-e} but Dee-lée-ge-re

{\bf S} is hard as in the English word {\it sea} but is slightly softwned when coming between two vowels.
e.g. {\it misericórdia.}

{\bf TI} standing before a vowel and following any letter (except {\bf S. X. T.}) is pronounved {\it tsee.}
e.g.
{\it Pa}{\sc ti}{\it én}{\sc ti}{\it a} = {\it Pa-t-see-én-see-a.}
{\it Grá}{\sc ti}{\it a} = {\it Grá-t-see-a.}
{\it Constitú}{\sc ti}{\it o} = {\it Con-stee-tú-t-see-o.}
{\it Laetí}{\sc ti}{\it a} = {\it Lae-tée-t-see-a.}

Otherwise the {\bf T} is like the English {\it T}.

{\it TH} always simply {\bf T.} {\it Thómas, cathólicam.}

{\bf X} is pronounced {\it ks,} slightly softened when coming between two vowels.
{\it e.g. exércitus.}

{\bf XC} before {\bf e, ae, oe, i, y = KSH}
e.g. {\it Excélsis = ek-shél-sees.}

Before other vowels {\bf XC} has the ordinary hard sound of the letters composing it.
e.g. {\bf KSC} {\it excussórum = eks-coos-só-room.}

{\bf Y} in Latin is reconed among the vowels and is sounded like {\bf I.}

{\bf Z} is pronouonced {\it dz. zizánia.}

All the rest of the consonants {\bf B, D, F, K, L, M, N, P, Q, V} are pronounced as in English.

Double Consants must be clearly sounded
e.g. {\it Bello = bel-lo,} not the English {\it bellow.}
Examples: {\it Abbas, Joánnem, Innocens, piíssime, terra.}

In the pronunciation and singing of a word the "Golden Rule" must always be kept"
"Never take breath just before a fresh syllable of a word".

XXX

A person who is unable to sing this phrase from the quarter-bar to the end in one breath, must be careful not to breathe just before a fresh syllable (at {\it a} or {\it b}). The lesser evil would be to breath {\it after} the long note and off its value:

XXX


