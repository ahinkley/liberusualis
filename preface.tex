\section{PREFACE TO THE VATICAN EDITION\\
\small{OF THE ROMAN CHANT.}}

{\it The place of honour in this Solesmes Edition of the Vatican Official text is given to the Vatican Preface. Its wise counsels and general Principles of interpretation are embodied, elucidated and enlarged upon in the Rules given further on.}

Holy Mother the Church has received from God the charge of training the souls of the faithful in all holiness, and for this noble end has ever made a happy use of the help of the sacred Liturgy. Wherein --- in order that men's minds may not be sundered by differences, but that, on the contrary, the unity which gives vigour and beauty to the mystical body of Christ might flourish unimpaired -- she has been zealous to keep the traditions of our forefathers, ever trying diligently to discover and boldly to restore any which might have been forgotten in the course of the ages.

Now among those things which most nearly touch the sacred Liturgy, being as it were interwoven therein and giving it splendour and impressiveness, the first place must be assigned to the Sacred Chant. We have, indeed, all learnt from experience that it gives a certain breadth to divine worship and uplifts the mind in wondrous wise to heavenly things. Wherefore the Church has never ceased to recommend the use of the Chant, and has striven with the greatest assiduity and diligence to prevent its decline from its pristine dignity.

To this end liturgical music must possess those characteristics which make it preeminently sacred and adapted to the good of souls. It must surely emphasise above all else the dignity of divine worship, and at the same time be able to express pleasantly and truly the sentiments of the christian soul. It must also be catholic, answering to the needs of every people, country and age, and combine simplicity with artistic perfection.

All these characteristics, however, are nowhere to be found in a higher degree than in Gregorian Chant -- {\it the special Chant of the Roman Church, who has received it alone by inheritance from the Fathers, has kept it carefully throughout the ages in her records, and commends it to the faithful as her own, ordering its exclusive use in certain parts of the Liturgy.} (Motu Proprio. Nov. 22. 1903. n. 3.)

Certainly in the course of time the Gregorian Chant incurred no small loss of purity. This was chiefly because the special rules of the Chant, as traditionally received from the Fathers, were either negligently overlooked or allowed to be altogether forgotten. Hence arose an evident decline in the spirit which is spoken of as "liturgical", and the "spirit of prayer", while at the same time the beauty and grace of the sacred melodies, if they did not wholly disappear, were certainly affected for the worse.

But the Sovereign Pontiff, Pius X. --- may his enterprise be crowned with good fortune and success! --- emulating herein the zealous endeavours of his predecessors, determined and took measures to prevent any further decadence in the Gregorian Chant. Wherefore, in his Motu Proprio, issued on November 22nd, 1903, he accurately and clearly laid down the principles (surely the first step of reform) whereon the ecclesiastical Chant is based and whereby it is controlled; he gathered together at the same time the principal regulations of the Church against the various abuses which had crept into the Chant in the course of time. And then appeared the Decree of the Congregation of Rites, issued on January 8th, 1904, wherein clearer directions were give for the restoration of the Gregorian Chant.

Nevertheless it remained for the Roman Church and the other Churches which follow her Rite, to provide themselves with books containing the true melodies of the Gregorian Chant. His Holiness, Pius X, had this in view when, in his Motu Proprio, promulgated on April 25th, 1904, he declared: the Gregorian melodies were to be restored in their integrity and identity, after the authority of the earliest manuscripts, taking account of the legitimate tradition of past ages, as well as of the actual use of the Liturgy of to-day.

Guided by these rules and standards, those who had taken the task in hand at the bidding of the Pope set to work to revise the books then in use. The first things they had to do was to undertake a thorough and well considered examination of the primitive manuscripts. This procedure was clearly a wise one; for documents of this kind are not merely to be esteemed on account of their antiquity, which unites them so closely to the beginnings of the Gregorian Chant, but chiefly because they were written in the very ages in which the Chant was most flourishing. For although the more remote the origin of the melodies and the longer they have been in use amongst the ancients, the more worthy they might be of finding a place in the new edition which was in hand, nevertheless, what gives them the right of being included is their religious an artistic flavour, and their power of giving suitable expression to liturgical prayer.

Therefore, in studying the manuscripts, thus was the primary object which was kept in view: not indeed to admit off-hand, on the sole ground of antiquity, whatever happened to be most ancient, but, since the restoration of the ecclesiastical Chant had to depend not only on paleographical considerations, but also was to draw upon history, musical and Gregorian art, and even upon experience and upon the rules of the sacred liturgy, it was necessary to have regard to all of these things at the same time; lest a piece, composed perhaps with the learning of antiquity, should fall short in some of the other conditions, and do injury to Catholic tradition by depriving many centuries of the right of contributing something good, or even better than itself, to the patrimony of the Church. For it is by no means to be admitted that what we call the Gregorian tradition may  be confined within the space of a few years; but it embraces all those centuries which cultivated the art of the Gregorian Chant with more or less zeal and proficiency. {\it The Church,} says the Holy Father in the Motu Proprio already mentioned, {\it as cultivated and fostered the progress of the arts unceasingly, allowing for the use of religion all things good and beautiful discovered by man in the course of the ages, provided that liturgical rules be observed.}

The work of the present edition has been carried out in accordance with these wise directions delivered by Our Most Holy Lord Pope Pius X.

The Church certainly gives freedom to all the learned to settle the age and condition of the Gregorian melodies, and to pass judgment upon their artistic skill. She only reserves to herself one right, to wit, that of supplying and prescribing to the Bishops and the faithful such a text of the sacred Chant as may contribute to the fitting splendour of divine worship and to the edification of souls, after being restored according to the traditional records.

Enough has already been said above to show how solidly based was the work so wisely undertaken of restoring the ancient and legitimate melodies of the Church to their integrity. But for the convenience of those who will be using the choir books edited in accordance with that has already been laid down, it is well to add here a few remarks about the proper notes and figures of the Gregorian Chant as well as about the right way of interpreting them.

For the proper execution of the Chant, the manner of forming the notes and of linking them together, established by our forefathers and in constant and universal use in the Middle Ages, is of great importance and is recommended still as the norm for modern Editors. The following tables give the principal forms of these notes or neums along with their names:\\

{\setlength{\parindent}{-0.5in} \parbox{5in}{
\begin{longtable}{C{1.22in}C{1.22in}C{1.22in}C{1.22in}}
{\centering \glyphsnippet{0}{.1}{i}} & {\centering \glyphsnippet{0}{.1}{jv}} & {\centering \glyphsnippet{0}{.1}{jvv}} & {\centering \glyphsnippet{0}{.1}{I}} \\[-0.35in]
Punctum \newline & Virga \newline & Bivirga \newline & \newline Punctum inclinatum \newline {\it (Diamond)}\\[0.3in]
{\centering \glyphsnippet{0}{.1}{hj}} & {\centering \glyphsnippet{0}{.1}{ji}} & {\centering \glyphsnippet{0}{.1}{ij~}} & {\centering \glyphsnippet{0}{.1}{ji~}} \\[-0.45in]
Podatus {\it or} Pes & Clivis {\it or} Flexa & Epiphonus & \newline \newline Cephalicus\\[0.3in]
{\centering \glyphsnippet{0}{.2}{ij!kv}} & {\centering \glyphsnippet{0}{.4}{hhi h/ij}} & {\centering \glyphsnippet{0}{.2}{jvIH}} & {\centering \glyphsnippet{0}{.4}{jV!ih~ jvIH}} \\[-0.45in]
Scandicus & Salicus & Climacus & \newline \newline Ancus\\[0.3in]
{\centering \glyphsnippet{0}{.2}{hih}} & {\centering \glyphsnippet{0}{.2}{jij}} & {\centering \glyphsnippet{0}{.2}{ijij}} & {\centering \glyphsnippet{0}{.2}{jijh}} \\[-0.45in]
Torculus & Porrectus & Torculus resupinus & \newline \newline Porrectus flexus\\[0.3in]
{\centering \glyphsnippet{0}{.2}{hjIH}} & {\centering \glyphsnippet{0}{.2}{hi!jvIH}} & {\centering \glyphsnippet{0}{.2}{hi/jh}} & {\centering \glyphsnippet{0}{.2}{jvIHi}} \\[-0.45in]
Pes subpunctis & Scandicus subpunctis & Scandicus flexus & \newline \newline Climacus resupinus\\[0.3in]
{\centering \glyphsnippet{0}{.5}{jj jjj}} & {\centering \glyphsnippet{0}{.2}{ij/j}} & {\centering \glyphsnippet{0}{.2}{jii}} & {\centering \glyphsnippet{0}{.2}{hihh}} \\[-0.45in]
Strophicus & Pes strophicus & \newline \newline Clivis strophica \newline {\it or} Clivis \newline {\it with an} Oriscus & \newline \newline Torculus strophicus {\it or} Torculus {\it with an} Oriscus\\
\end{longtable}
\begin{longtable}{C{1.22in}C{2in}C{1.5in}}
{\centering \glyphsnippet{0}{.2}{jji}} & {\centering \glyphsnippet{0}{.2}{jiih}} \hspace{0.2in} {\centering \glyphsnippet{0}{.2}{jiihi}} \hspace{0.2in} {\centering \glyphsnippet{0}{.2}{jvIHhg}} \hspace{0.2in} {\centering \glyphsnippet{0}{.3}{hj!kvkvIH}} & {\centering \glyphsnippet{0}{1}{jjh  h!jjh  jjvHG}} \\[-0.4in]
Pressus & {\it Other} Pressus {\it or apposed neums} & \newline \newline Trigon\\[0.3in]
\end{longtable}
\vspace{-0.45in}
\begin{longtable}{C{2in}C{2in}}
{\centering \glyphsnippet{0}{.1}{fw}} \hspace{0.1in} {\centering \glyphsnippet{0}{.2}{f!gwh}} \hspace{0.1in} {\centering \glyphsnippet{0}{.2}{f!gwhg}} \hspace{0.1in} {\centering \glyphsnippet{0}{.3}{f!gw!hvGF}} &
{\centering \glyphsnippet{0}{.2}{highf}} \hspace{0.3in} {\centering \glyphsnippet{0}{.2}{f!gh!iv}} \hspace{0.2in} {\centering \glyphsnippet{0}{.2}{f!ghGFg}} \hspace{0.3in} {\centering \glyphsnippet{0}{.3}{hghGFg}} \\
Quilisma & {\it Longer or compound Neums}
\end{longtable}
}}

%{\centering \glyphsnippet{0}{0.2}{hw}} & {\centering \glyphsnippet{0}{0.2}{h!iwj}} & {\centering \glyphsnippet{0}{0.2}{h!iwji}} & {\centering \glyphsnippet{0}{0.2}{h!iw!jvIH}} & \\
%{\centering \glyphsnippet{0}{0.2}{highf}} & {\centering \glyphsnippet{0}{0.2}{f!gh!iv}} & {\centering \glyphsnippet{0}{0.2}{h!ijIHi}} & {\centering \glyphsnippet{0}{0.2}{jijIHi}} & \\

To avoid all error and doubt in the interpretation of the above notation, the following observations are to be noted:

1. Of the two notes of the {\it Podatus}, the lower note must be sung before the upper note immediately above it.\\
\centersnippet{4cm}{(c4) () fa {}sol(fg) (:) ré {}la(ch) (:) sol {}ut(gj) (::)}

2. The heavy slanting line of the {\it Porrectus} stands for the two notes which it links to gether, so that the first note is given at the top of the line and the lower note at the lower end of the line:

\centersnippet{3.5in}{(c4) () la {}sol la(hgh) (:) la {}fa sol(hfg) (:) sol {}mi sol(geg) (:) fa s{o}l re mi(fgde) (:)}

3. The half-note, which terminates the {\it Cephalicus} \glyphsnippet{-5}{0.1}{gf~} and the {\it Epiphonus} \glyphsnippet{-5}{0.1}{fg~}, only occurs at the end of a syllable when the next syllable leads on to the combination of two vowels like a diphthong. as e.g. {\sc au}{\it tem}, {\sc ei}{\it us}, {\it allel}{\sc ui}{\it a} : or cases the nature of the syllables obliges the voice, in passing form one to the other, to flow or become "liquescent", so that, being confined in the mouth, it does not seem to end, but to lose half its force rather than its duration. (Cf. Guido. {\it Microl}. Cf. XV.)

When, however, the nature of the syllables requires a sound which is not liquescent but emitted in full, the {\it Epiphonus} becomes a {\it Podatus}, the {\it Cephalicus} a {\it Clivis}:

\gabcsnippet{(c3) <alt>Epiphonus</alt> A(d) sum(ef~)mo.(f) (:) <alt>Podatus</alt> In(d) so(ef)le.(f) (::) (c4) <alt>Cephalicus</alt> Te(gh) lau(ji~)dat.(jk) (:) <alt>Clivis</alt> Sol(gh)vé(ji)bant.(jk) (::)}

It sometimes happens that two notes follow another higher note or {\it Virga} in the manner of a {\it Climacus;} they may be liquescent, at any rate the last of them. In this case they are represented by two diamond shaped notes of smaller size \glyphsnippet{-5}{.2}{gvFE}, or they are changed into a {\it Cephalicus} following the {\it Virga} \glyphsnippet{-5}{.15}{g!fe~}. This kind of neum, which is akin to the {\it Climacus,} is called an {\it Ancus.}

4. When several simple notes as in the {\it Strophicus} or the {\it Pressus} or the like are in apposition, that is to say, so written on the same line as to be near one another, they must be sustained for a length of time in proportion to their number. There is, however, this difference between the {\it Strophicus} and the {\it Pressus,} that the latter should be sung with more intensity, or even, it it be preferred, {\it tremolo;} the former more softly, unless the tonic accent of the corresponding syllable require a stronger impulse.

5. There is another kind of {\it tremolo} note, i.e. the {\it Quilisma,} which appears in the chant like a "melodic blossom". It is called a {\it "nota volubilis"} and {\it "gradata"}, a note with a trill and gradually ascending. If one has not learnt how to execute these {\it "tremolo"} or shaken notes, or, knowing how to render them, has nevertheless to sing with others, he should merely strike the preceding note with a sharper impulse so as to refine the sound of the {\it Quilisma} rather than quicken it.

6. The tailed note which marks the top of the {\it Climacus}, {\it Clivis} and {\it Porrectus,} is a distinguishing characteristic of these neumatic forms as they have been handed down by our forefathers. This particular note often receives a stronger impulse, not because it is tailed, but because it is not joined to any preceding note, and therefore it gets a direct vocal impulse. The little line which is sometimes drawn from one note to the next merely serves to bind the two together.

7. In themselves the descending diamond notes, which in certain neums follow the culminating note, have no {\it special} time-value \glyphsnippet{-5}{.2}{gvFE}. Their peculiar form and their slanting arrangement show their subordination to the cuilminating note, and must therefore be rendered by connecting the notes together.

Single neums, however their constituent parts may be combined in the writing, are to be sung as a single whole, in such wise that the notes which follow the first may appear to spring from it, making all the notes rise and flow from a single vocal impulse.

The reason which demands the joining together of the notes of the same neum, both in the musical tect and in the singing of it, also requires that the neums should be marked off from one another alike for the eye and for the ear: and this is done in various ways according to the various contexts.

1. When several neums correspond with several syllables, and the syllables are separately articulated, the neums are thereby divided. Then the neum adapted to each syllable changes its quality and strength by receiving a stronger accent if the syllable to which it belongs is strongly accented, but it is weaker if the nature of the corresponding syllable needs less emphasis.

2. When several neums are adapred to the same syllable, then the whole series is so divided into parts that some flow on almost, or altogether, linked to one another (see A below): whereas others are separated by a wider interval (B), or by a dividing line (C), and are sustained by a slight {\it ritenuto} of the voice {\it (mora vocis)} at the final syllable, a slight breathing being permitted if required:\\

{\setlength{\parindent}{-0.2457 cm} \parbox{4.8in}{\gabcsnippet{(c4) () Ký(h)ri(jk)e<alt> \hspace{0in} D \hspace{0in} B \hspace{0.38in} A</alt> (kv//jkh//ixgh!ivHGF/ghh) <alt>\hspace{-3pt}C</alt>*(,) <alt> \hspace{0.05in} D \hspace{0.12in} B \hspace{0.25in} A</alt>(hj!kv//jkh//ixgh!ivHGF/ghh) <alt>\hspace{-3pt}C</alt>**(,) <alt> \hspace{0.1in} A \hspace{0.17in} B</alt>(hvGF/gvFED//fvDCd) e(cd)lé(fg)i(f)son.(ed) (::)}}}
\vspace{10pt}

Observe that a tailed note, (D), immediately followed by a neum which it commands does not indicate a breathing but rather a longer pause.

According to the "golden rule", there must be no pause at the end of any neum followed immediately by a new syllable of the same word; by no means must there be a lengthening of sound still less a silent beat, for this would break up and spoil the diction.

In every piece of chant such divisions must be observed as the words or melodies require or allow. To assist singers various signs of musical punctuation are already in use in Chant books, according to the kind or extent of the various divisions or pauses: v. g.

\vspace{10pt}
{\small \hspace{0.03in} 1. Major division. \hspace{0.07in} 2. Minor division. \hspace{0.4in} 3. Small division. \hspace{0.333in} 4. Final division.}
\gabcsnippet{() <v>\hspace{0.5in}</v>() (:) <v>\hspace{1.1in}</v>() (;) <v>\hspace{1.1in}</v>() (,) <v>\hspace{1in}</v>() (::)}
\vspace{10pt}

1. A major division or pause, also called a dividing pause, is made by giving a greater prolongation to the last notes and by taking a full breath.

2. A minor pause, or subdivisional pause, requires a lesser prolongation, and gives time for a short breath.

3. A brief pause or small division indicates a short sustaining of the voice, and permits, if necessary, the taking of a very short breath. Should the singer require to take breath at more frequent intervals, he may snatch one whereever the words or music allow an interstice, but he must nevere make any break in the words or neums themselves.

4. A double line closes either a piece of the Chant or one of its principal parts.

In books of Chant another {\it r\^ole} is also assigned to this double line: for it is used in addition to mark the place where, after teh beginning, the whole choir takes up the singing, or where the chanting alternates and changes sides. But since this sort of sign incorporated in the midst of the musical text often does injury to the coherence of the Chant, it has been thought more fitting to replace it with an asterisk *, as may be seen in the above example of the {\it Kyrie eleison.} There, and in similar places, a single asterisk will be found, to show that one side of the choir is to be followed by the other side singing alone; but a double asterisk ** will be seen where the full choir ought to take up the Chant, so as to end, as is rightm with the combined voices of the whole choir.

It is to be noted that B-flat, when it occurs, only holds good as far as the next natural ({\makeatletter\gre@font@music\GreCPNatural}), or dividing line, or new word.

When these points have been thoroughly understood, those who take part in divine worship should also learn all the rules of the Chant and be diligent in their observance, but in such a way that their mind is ever in accord with their voice.

First of all, care should be taken that the words to be sung are clearly and thoroughly understood. (Benedict XIV). For the Chant ought not to weaken but to improve the sense of the words. (St Bernard. {\it Ep. 312}).

In all texts, whether of lessons, psalmody or chants, the accent and rhythm of the word are to be observed as far as possible, for thus it is, that the meaning of the text is best brought out. {\it (Instituta Patrum.)}

Moreover, great care must be taken not to spoil the sacred melodies by enevenness in the singing. No neum or note should ever be unduly shortened or prolonged. The singing must be uniform, and the singers should listen to one another, making their pauses well together. When the musical movement is slower, the pause must be lengthened. In order that all the voices may be one, which is most essential, each singer should attempt in all modesty to allow his own voice to become merged in the volume of sound of the choir as a whole. Neither are those to me imitated who hurry the Chant thoughtlessly or who drag out the syllables heavily. But every melody, whether it be sung slowly or quickly, must be executed with fluency, roundness and in a melodious manner. (Hucbald. Nicetas. {\it Instit. Patrum.})

The above rules have been drawn from the holy Fathers, some of whom learnt this way of singing from the Angels, while others received it from the teaching of the Holy Spirit speaking to their hearts in contemplation. If we set ourselves to practice these principles with diligence, we too shall appreciate the subtle charm of the Chant, singing to God in our heart and spirit and mind. ({\it Instituta Patrum.})

Moreover, those whose duty it is to sing in the Church of God must also be well instructed in the rubrics of their office. Wherefore the principal rules with reference to the Gradual are given below.
